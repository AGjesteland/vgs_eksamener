% TITTEL: Eksamen i matematikk S2, våren 2017
% BESKRIVELSE: Et stilrent dokument som samsvarer noe
%              med stilen til en eksamen, men med
%              Computer Modern skrift og noen andre forskjeller.
% FORFATTER: Tommy O.
% LISENS: Alle kan endre og bruke denne .TeX malen

%-----------------------------------------------------------
% Importering av pakker
%-----------------------------------------------------------
\documentclass[12pt, a4paper]{article}
\usepackage[utf8]{inputenc}
\usepackage[norsk]{babel}
\usepackage{amsmath, amsthm, amsfonts, amssymb}
\usepackage{mathtools}
\usepackage{hyperref}
\usepackage{fancyhdr}
\usepackage[margin = 3cm]{geometry}
\usepackage[sharp]{easylist}
\usepackage{lastpage}
\usepackage{parskip}

%-----------------------------------------------------------
% Oppsett av pakker
%-----------------------------------------------------------
\rfoot{\small Side \thepage\ av \pageref*{LastPage}}
\cfoot{}
\lfoot{\small Eksamen REA3028 Matematikk S2 Våren 2017}
\renewcommand{\headrulewidth}{0pt}
\renewcommand{\footrulewidth}{0.5pt}

%-----------------------------------------------------------
% Start av selve dokumentet
%-----------------------------------------------------------

\begin{document}
	\pagestyle{fancy}
	{\bfseries \Large Eksamen S2, våren 2017} \\
	{ \footnotesize Laget av Tommy O. -- Sist oppdatert: \today}
	\hrule
	\footnotesize 
	\textbf{Kommentar:}
	Dette er en innskriving av S2 eksamen,
	basert på scan av dokumentet lastet opp
	av \texttt{matematikk.net}-bruker Viks.
	Det er en viktig læringsprosess å kunne regne
	gjennom tidligere eksamener.
	Dersom du selv er elev/privatist, delta på diskusjon på
	\texttt{matematikk.net} om eksamener og last gjerne
	opp nyttig materiale på forumet.
	\normalsize
	
	\hrule
	
\section*{Del 1 - uten hjelpemidler}
%-----------------------------------------------------------
\vspace*{2em}
{\bfseries \large Oppgave 1} (5 poeng) \vspace*{1em} \\
Deriver funksjonene
\begin{easylist}[enumerate]
	\ListProperties(Numbers1=l, FinalMark={)})
	# $f(x) = x^2 - 2/x$
	# $g(x) = \ln \left(x^2 + 1\right)$
	# $h(x) = x^2 e^x$
\end{easylist}
%-----------------------------------------------------------
\vfill
\vspace*{2em}
{\bfseries \large Oppgave 2} (2 poeng) \vspace*{1em} \\
Løs likningssystemet
\begin{align*}
x + y -z &= 0 \\
2x + y -z &= 2 \\
4x + y -2z &= 1 
\end{align*}
%-----------------------------------------------------------
\vfill
\vspace*{2em}
{\bfseries \large Oppgave 3} (6 poeng) \vspace*{1em} \\
I en aritmetisk rekke $a_1 + a_2 + a_3 + \dots + a_n$ 
er $a_1 = 3$ og $a_6 = 18$.
\begin{easylist}[enumerate]
	\ListProperties(Numbers1=l, FinalMark={)})
	# Bestem differansen $d$, og bestem en formel for $a_n$
	uttrykt ved $n$.
	# Vis at summen av de $n$ første leddene kan skrives som
	$$S_n = \frac{3}{2}n (n+1)$$
	# Hvor mange ledd må vi ha med for at summen skal bli 84?
\end{easylist}
\vfill
%-----------------------------------------------------------
\clearpage
{\bfseries \large Oppgave 4} (7 poeng) \vspace*{1em} \\
Funksjonen $f$ er gitt ved
$$f(x) = x^3 + 4x^2 + x - 6$$
\begin{easylist}[enumerate]
	\ListProperties(Numbers1=l, FinalMark={)})
	# Vi ser at $f(1) = 0$.
	Bruk blant annet polynomdivisjon til å vise at
	$$f(x) = (x-1)(x+2)(x+3)$$
	# Løs ulikheten $f(x) \leq 0$
	# Forkort brøken mest mulig
	$$
	\frac{x^3 + 4x^2 + x - 6}{2x^2 - 2}
	$$
	# Bruk blant annet det du viste i oppgave a), til å løse likningen
	\begin{equation*}
		e^{3x} + 4e^{2x} + e^x - 6 = 0
	\end{equation*}
\end{easylist}
%-----------------------------------------------------------
\vfill
\vspace*{2em}
{\bfseries \large Oppgave 5} (6 poeng) \vspace*{1em} \\
Totalkostnaden i kroner ved produksjon av en vare er gitt ved
\begin{equation*}
	K(x) = 0.1x^2 + 70x + 4000 \quad , \quad 0 < x < 2000
\end{equation*}
Her er $x$ antall produserte enheter per uke.

Inntekten i kroner ved denne produksjonen er gitt ved
\begin{equation*}
	I(x) = -0.05x^2 + 280x \quad , \quad 0 < x < 2000
\end{equation*}
\begin{easylist}[enumerate]
	\ListProperties(Numbers1=l, FinalMark={)})
	# Bestem $K'(500)$ og $I'(500)$.
	Bruk svarene til å vurdere om bedriften bør produsere flere enn 500 enheter.
	
	# Bestem den vinningsoptimale produksjonsmengden, det vil si den produksjonsmengden som gir størst overskudd.
	
	# Bestem den kostnadsoptimale produksjonsmengden, det vil si den produksjonsmengden som gir lavest kostnad per enhet.
\end{easylist}
\vfill
%-----------------------------------------------------------
\clearpage
{\bfseries \large Oppgave 6} (10 poeng) \vspace*{1em} \\
La $X$ være antall unger som overlever i en tilfeldig
valgt fuglekasse med kjøttmeis. Sannsynlighetsfordelingen
til $X$ er gitt i tabellen nedenfor.

\begin{center}
	\begin{tabular}{|c|c|c|c|c|c|}
		\hline
		$k$ & 0 & 1 & 2 & 3 & 4 \\ \hline
		$P(X = k)$ & 0.2 & 0.1 & 0.3 & 0.3 & 0.1 \\ \hline
	\end{tabular}
\end{center}
\begin{easylist}[enumerate]
	\ListProperties(Numbers1=l, FinalMark={)})
	# Bestem $P(X \geq 2)$.
	# Bestem forventningsverdien $\operatorname{E}(X)$,
	og vis at standardavviket er $\operatorname{SD}(X) = \sqrt{1.6}$\\
	Hva forteller $\operatorname{E}(X)$ oss?
\end{easylist}

Et år har biologilærer Peder overvåket 100 
fuglekasser med kjøttmeis. Kassene er nummerert fra 1 til
100. La $X_i$ være antall kjøttmeisunger som overlever i kasse
nummer $i$. Vi antar at $X_i$-ene er uavhengige. 
Det totale antallet kjøttmeisunger som overlever i de 100
kassene, er gitt ved den stokastiske variabelen
\begin{equation*}
	S = X_1 + X_2 + \dots + X_{100}
\end{equation*}

\begin{easylist}
	\ListProperties(Numbers1=l, FinalMark={)})
	# Begrunn at $S$ er tilnærmet normalfordelt.
	# Bestem $\operatorname{E}(S)$ og $\operatorname{Var}(S)$.
\end{easylist}

I resten av oppgaven går vi ut i fra at 
$\operatorname{E}(S) = 200$ og 
$\operatorname{SD}(S) = 13$.
Du vil få bruk for standard normalfordelingstabellen i vedlegg 1.

\begin{easylist}
	\ListProperties(Numbers1=l, FinalMark={)})
	# Bestem sannsynligheten for at 226 eller flere kjøttmeisunger
	overlever i kassene til Peder dette året.
	# Bestem $P(187 \leq S \leq 213)$.
\end{easylist}
\pagebreak
%-----------------------------------------------------------
\section*{Del 2 - med hjelpemidler}
%-----------------------------------------------------------
\vspace*{2em}
{\bfseries \large Oppgave 1} (6 poeng) \vspace*{1em} \\
En bedrift produserer en vare.
Bedriften selger alt den produserer.
Overskuddet $O$ ved salg av $x$ enheter per uke er gitt ved
\begin{equation*}
	O(x) = ax^2 + bx + c
\end{equation*}
\begin{easylist}[itemize]
	# Når bedriften produserer 200 enheter per uke,
	blir overskuddet lik 0.
	# Overskuddet er størst når bedriften
	selger 474 enheter.
	# Når bedriften selger 600 enheter per uke,
	er grensekostnaden 5 kroner større enn grenseinntekten.
\end{easylist}
\begin{easylist}[enumerate]
	\ListProperties(Numbers1=l, FinalMark={)})
	# Vis at disse opplysningene gir likningssystemet
	\begin{align*}
	 40000a + 200b + c &= 0 \\
	 950 a + b &= 0 \\
	 1200a + b &= -5
	\end{align*}
	
	# Bruk CAS til å bestemme $a$, $b$ og $c$.
	
	# Hva er det største overskuddet bedriften
	kan får per uke?
\end{easylist}
%-----------------------------------------------------------
\vfill
\vspace*{2em}
{\bfseries \large Oppgave 2} (6 poeng) \vspace*{1em} \\
Ingrid inngår en pensjonsavtale der hun skal spare 20 000
kroner i året. Den første innbetalingen skjer 1. januar 2018.
Den siste innbetalingen skjer 1. januar 2052.
Hun får en fast rente på 3.00 \% per år.
\begin{easylist}[enumerate]
	\ListProperties(Numbers1=l, FinalMark={)})
	# Bruk CAS til å vise at Ingrid vil ha
	1 209 242 kroner på konto rett etter
	siste innbetaling.
\end{easylist}
Ingrid planlegger å ta ut et fast beløp hvert år.
Det første uttaket vil hun gjøre 1. januar 2053 og
det siste uttaket 1. januar 2067. Da skal kontoen være tom.
Hun regner med en rente på 3.00 \% per år.
\begin{easylist}
	\ListProperties(Numbers1=l, FinalMark={)})
	# Hvor mye kan hun ta ut per år?
\end{easylist}
Ingrid synes det er tilstrekkelig å ta ut 80 000 kroner
i året fra og med 1. januar 2053.
\begin{easylist}
	\ListProperties(Numbers1=l, FinalMark={)})
	# Bruk CAS til å bestemme når kontoen 
	vil være tom i dette tilfellet.
\end{easylist}
\vfill
%-----------------------------------------------------------
\clearpage
{\bfseries \large Oppgave 3} (4 poeng) \vspace*{1em} \\
En bedrift produserer batterier til hodelykter.
Bedriften påstår at levetiden for batteriene
er 30 timer når hodelykten brukes med full lysstyrke.

Vi antar at levetiden for batteriene er normalfordelt
med $\mu = 30$ timer og $\sigma = 3$ timer.
\begin{easylist}[enumerate]
	\ListProperties(Numbers1=l, FinalMark={)})
	# Bestem sannsynligheten for at et tilfeldig valgt
	batteri har en levetid på mindre enn 27 timer.
\end{easylist}

Forbrukerrådet har mistanke om at forventet levetid er mindre
enn 30 timer. Derfor blir ni tilfeldig valgte batterier testet.
Levetida til batteriene viser seg å være
$29, 31, 32, 27, 29, 25, 23, 30$ og $26$ timer.

Vi antar at levetiden til batteriene er normalfordelt med
$\sigma = 3$ timer.

\begin{easylist}
	\ListProperties(Numbers1=l, FinalMark={)})
	# Sett opp en hypotesetest, og bruk den til å avgjøre
	om det er grunnlag for å hevde at den forventede levetiden
	er mindre enn 30 timer. Bruk et signifikansnivå på 5 \%.
\end{easylist}
%-----------------------------------------------------------
\vfill
\vspace*{2em}
{\bfseries \large Oppgave 4} (8 poeng) \vspace*{1em} \\
I et område er det brutt ut en smittsom sykdom.
Antall personer som blir smittet per uke,
kan modelleres med en logistisk funksjon $g$ der
\begin{equation*}
	g(t) = \frac{N}{1 + ae^{-kt}}
\end{equation*}
Her er $N$, $a$ og $k$ reelle tall, og $g(t)$ er antall personer
som blir smittet per uke, $t$ uker etter at sykdommen
brøt ut.

Tabellen nedenfor viser $g(t)$ for noen verdier av $t$.
\begin{center}
	\begin{tabular}{|c|c|c|c|}
		\hline 
		$t$ & 0 & 5 & 12 \\ \hline
		$g(t)$ & 200 & 2000 & 9000 \\ \hline
	\end{tabular}
\end{center}

\begin{easylist}[enumerate]
	\ListProperties(Numbers1=l, FinalMark={)})
	# Bruk regresjon til å bestemme $N$, $a$ og $k$ i uttrykket
	$g(t)$.
\end{easylist}
Nærmere undersøkelser viser at
\begin{equation*}
f(t) = \frac{10000}{1 + 50e^{-0.5t}}
\end{equation*}
er en god modell for antallet som blir smittet per uke, $t$
uker etter at sykdommen brøt ut.
\begin{easylist}
	\ListProperties(Numbers1=l, FinalMark={)})
	# Bruk graftegner til å tegne grafen til $f$.
	Bruk grafen til å bestemme når antall smittede
	personer per uke er 7000.
	# Bruk CAS til å bestemme $\int_{0}^{12} f(t) \ dt$.
	Hva forteller dette svaret oss?
	# Hvor mange uker vil det gå før det totale antallet 
	personer som er smittet, overstiger 15 000?
\end{easylist}
\vfill
\end{document}